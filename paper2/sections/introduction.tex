\newpage
\section{Introduction}
\lettrine[findent=2pt]{\fbox{\textbf{C}}}{ }\textit{atalan Numbers} were first discovered by Swiss mathematician \textit{Leonhard Euler} and Hungarian mathematician \textit{Johann Andreas von Segner} by studying the problem of triangulation of the convex polygon. 
Although recursive relations of these numbers are introduced from Segner, many of the properties and identities of these numbers are discovered by the side of French-Belgian mathematician Eugene Charles Catalan in 1838 through the study of well-formed sequences of parentheses.

Catalan numbers have a significant place and major importance in combinatorics and computer science. They form a sequence of natural numbers that occur in studying astonishingly many combinatorial problems. In mathematics, Catalan numbers describe the number of ways a polygon with 𝑛+2 sides can be cut into 𝑛 triangles, the number of rooted, trivalent trees with n+1 nodes, the number of paths of length 2n through an n×n grid that do not rise above the main diagonal. 