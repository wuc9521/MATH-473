\newpage
\section{Introduction}
\lettrine[findent=2pt]{\fbox{\textbf{C}}}{ }\textit{atalan Numbers} is a kind of counting sequence which was first discovered by Swiss mathematician \textit{Leonhard Euler} and Hungarian mathematician \textit{Johann Andreas von Segner}\cite{oconnor2004segner} by studying the problem of triangulation of the convex polygon\cite{mathshistory}. 
Although recursive relations of these numbers are introduced from Segner, many of the properties and identities of these numbers are discovered by the side of French-Belgian mathematician Eugene Charles Catalan in 1838 through the study of well-formed sequences of parentheses.

We choose this topic, because it is originally covered in our textbook, but removed in the revised version. We believe this topic is still interesting and important in Computational Mathematics. 