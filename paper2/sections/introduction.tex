\newpage
\section{Introduction}
\lettrine[findent=2pt]{\fbox{\textbf{C}}}{ }\textit{atalan Numbers} is a kind of counting sequence which was first discovered by Swiss mathematician \textit{Leonhard Euler} and Hungarian mathematician \textit{Johann Andreas von Segner} (\cite[see][]{oconnor2004segner}), who study the problem of triangulation of the convex polygon (\cite[see][]{mathshistory}). 
Although recursive relations of these numbers are introduced from Segner, many of the properties and identities of these numbers are discovered by the side of French-Belgian mathematician Eugene Charles Catalan in the study of well-formed sequences of parentheses. 

Catalan published his study of Catalan Numbers, Solution d'un problème de Probabilité relatif au jeu de rencontre, in the second volume of Liouville’s Journal de Mathématiques Pures et Appliquées in 1837. Later, Catalan published some further study of Catalan Numbers in the same journal in 1838 and 1839.

This topic is originally covered in the Chapter 25 ``Combinatorics'' of 3rd edition textbook but removed in the revised version. The reason for choosing it is that we still believe this topic is interesting and important in Computational Mathematics. Also, this topic includes what we learned in week 04/01: using power series to expand the generating function (formula~\ref{generating-function}).