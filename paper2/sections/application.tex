\section{Application}

A typical application of Catalan Numbers is counting the number of ways starting from $(0, 0)$ to $(n, n)$ in a $n \times n$ grid without crossing the line $y = x-1$  (figure \ref*{fig:symmetry}).


The general formula and the recurrence formula can be interpreted through this application. The general formula is used to directly calculate the number of ways to reach $(n, n)$ from $(0, 0)$ in a $n \times n$ grid without crossing the line $y = x-1$. The recurrence formula is used to calculate the number of ways to reach point $(n, n)$ from $(0, 0)$ in a $n \times n$ grid without crossing the line $y = x-1$ by using previous number derived from point $(n-1, \ n-1)$.

\begin{multicols}{2}
    \begin{figure}[H]
        \centering
        \includegraphics[width=0.48\columnwidth]{figures/symmetry.pdf}
        \caption{general Formula}\label{fig:symmetry}
    \end{figure}
    \begin{figure}[H]
        \centering
        \includegraphics[width=0.56\columnwidth]{figures/recurrence.pdf}
        \caption{recurrence Formula}\label{fig:recurrence}
    \end{figure}
\end{multicols}


For general formula, it is clear that the number of ways to reach $(n, n)$ from $(0, 0)$ in a $n \times n$ grid without crossing the line $y = x-1$ is equal to the total number of ways to reach $(n, n)$ from $(0, 0)$ minus the number of ways to reach $(n, n)$ from $(0, 0)$ in a $n \times n$ grid crossing the line $y = x-1$. The Figure~\ref{fig:symmetry} shows to do a symmetry transformation with respect to the line $y = x-1$. Then

\begin{equation}
    C_n = \binom{2n}{n} - \binom{2n}{n-1} = \frac{1}{n+1}\binom{2n}{n}.
\end{equation}

For recurrence relation, suppose a point in the moving process first touches the line $y=x$ at point $(k,k)$, then the path can be divided into two parts: the first part is from $(0,0)$ to $(k,k)$, and the second part is from $(k,k)$ to $(n,n)$. By definition, the second part will be $C_{n-k}$ ways. And for first part, because next step of $(0,0)$ must be $(0,1)$, and the former step of $(k,k)$ must be $(k-1,k)$. So, the first step can be changed to $(0,1)$ to $(k-1,k)$. And we build new coordinate system with $(0,1)$ be the new origin, $(k-1,k)$ becomes $(k-1,k-1)$. Then, the first step by definition will be $C_{k-1}$. By principle of multiplication, it will be in $C_{k-1}$  $C_{n-k}$ ways. Therefore,

\begin{equation}
    \begin{aligned}
        C_{n}& =\sum_{k=1}^nC_{k-1}C_{n-k}  \\
        &=\sum_{k=0}^{n-1}C_{(k+1)-1}C_{n-(k+1)} \\
        &=\sum_{i=0}^{n-1}C_iC_{n-i-1}
    \end{aligned}
\end{equation}