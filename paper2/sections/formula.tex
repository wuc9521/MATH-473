\section{Recurrence Relation}
The definition of Catalan Numbers is given by the following formula
\[
    \begin{aligned}
        C_n = \frac{1}{n+1}\binom{2n}{n} = \frac{(2n)!}{(n+1)!n!}.\quad n \geq 0.
    \end{aligned}
\]

or is more commonly defined by the following recurrence relation:

\[
    \begin{aligned}
        C_0     & = 1,                                          \\
        C_{n+1} & = \sum_{i=0}^{n} C_i C_{n-i}, \quad n \geq 0.
    \end{aligned}
\]

These two definitions are proved equivalent by using the method of generating functions.
Suppose there is a function $g(x)$ such that
\[
    g(x) = \sum_{n=0}^{\infty} C_n x^n.
\]
where $C_n$ is the $n$-th Catalan number. Then we have~\footnote[1]{Note that in Combinatorics, we don't really care about whether this series converges, we always assume that it will converge in a certain radius.}

\[
    \begin{split}
        \begin{aligned}
            g^2(x)
             & = \sum_{n=0}^{\infty} \left(\sum_{i=0}^{n} C_i C_{n-i}\right) x^n. \\
             & = \sum_{n=0}^{\infty} C_{n+1} x^n.                                 \\
             & = \frac{g(x) - 1}{x}.
        \end{aligned}
    \end{split}
    \quad \stackrel{g(0)=1}{\Longrightarrow} \quad
    \begin{split}
        \begin{aligned}
            g(x)
             & = \frac{1 - \sqrt{1 - 4x}}{2x}                               \\
             & = -\frac{1}{2x}\sum_{n=1}^{\infty}\binom{1 / 2}{n}(-4 x)^{n} \\
             & = \sum_{n=0}^{\infty} \frac{1}{n+1}\binom{2n}{n}x^n.
        \end{aligned}
    \end{split}
\]


\noindent By applying the Taylor Expansion, we can derive that these two definitions are equivalent as the coefficient of $x^n$ in the series expansion of $g(x)$ is $C_n$.