\section{Research of Haskell Curry}

Like many mathematical logicians of his time, Curry focused on foundational issues in mathematics, aiming to establish a secure theoretical foundation for the discipline. In 1933, Curry learned of the \textbf{Kleene-Rosser Paradox}\cite{kleene-rosser-paradox} through correspondence with \textit{John Rosser}\cite{j-barkley-rosser}, another notable mathematical logician. This paradox, discovered jointly by Kleene and Rosser, demonstrated the incompleteness of several formal systems proposed by Alonzo Church and Curry himself. Despite the completion of this proof, Church, Kleene, and Rosser abandoned further research into the foundations of mathematics, whereas Curry persevered, declaring himself a ``\textbf{deserter never of paradoxes}''.

Curry placed all his hopes for the \textbf{Foundations of Mathematics} on \textbf{Combinatory Logic}, dedicating his life's work to its development. 
Although the concept of combinatory logic was initially proposed by Moses Schönfinkel, much of the research and advancements came from Curry, earning him recognition as one of its founders. 
From a modern perspective, \textbf{Combinatory Logic} and $\lambda$\textbf{-Calculus}\cite{lambda-calculus} together form the theoretical basis of \textbf{Functional Programming Languages}, with $\lambda$-Calculus pioneered by Alonzo Church\cite{alonzo-church}. 
However, Church's reputation and the fame of $\lambda$-Calculus overshadowed Curry and combinatory logic for a considerable period.

Curry's greatest contribution, and the focus of his lifelong research, is combinatory logic. In addition to this, he also discovered the Curry Paradox, named after him, and the famous \textit{Curry-Howard Isomorphism}, also known as the \textit{Curry-Howard Correspondence}. Moreover, his name is used as a verb — \textit{currying} — which refers to the process of transforming a multi-parameter function into a series of single-parameter functions.