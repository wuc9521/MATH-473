\newpage
\section{Introduction}
\lettrine[findent=2pt]{\fbox{\textbf{H}}}{ }\textit{askell Brook Curry} —— an American mathematician and logician, is a pivotal figure in the history of \textbf{Mathematical Logic} and \textbf{Computer Science}. 
While not as widely celebrated as \textit{Alan Turing}\cite{alan-turing} nor as extensively influential as \textit{Kurt Gödel}\cite{godel}, Curry's contributions to humanity in the real historical context can rival both. 
If the Turing Machine\cite{turing-machine} is considered the fundamental model of modern computer programming languages, then the notion of Computable Functions\cite{computable-function} represents another equivalent model alongside the Turing machine. 
Combinatory logic\cite{combinational-logic}, the result of Curry's lifelong dedication, has become a shining star in theoretical computer science research, with functions as its fundamental elements. 
Perhaps unbeknownst to many, Curry stands as the only mathematician and logician whose entire name has been used in computer language nomenclature: 
\begin{itemize}[label={}, leftmargin=1em, itemsep=0.5em]
    \item \textbf{Haskell}: the most popular functional programming language today; 
    \item \textbf{Brook}: a stream processing programming language developed at Stanford University, geared towards graphics processing and parallel computing, designed based on the ANSI C standard; 
    \item \textbf{Curry}: represents a novel language - a blend of functional and logical paradigms, with its foundation rooted in Haskell. In a sense, Curry can be seen as a superset of Haskell. 
\end{itemize}

\noindent Haskell Brook Curry, his works along with his name have become symbols of human intellect.
