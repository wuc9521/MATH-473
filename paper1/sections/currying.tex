\section{Currying and an Example}
In mathematics and computer science, currying\cite{currying} is the technique of translating the evaluation of a function that takes multiple arguments (or a tuple of arguments) into evaluating a sequence of functions, each with a single argument.
Since the logic part requires too much mathematical knowledge that I could barely cover them in 1500 words, I think using some simple programming codes here can make it easier to explain the main work of currying.

I'll firstly show the direct definition of currying:
given a function 
\[
    f: (X \times Y) \rightarrow Z    
\]
currying constructs a new function
\[
\begin{aligned}
    g: X \rightarrow (Y \rightarrow Z) \\
    (x)(y) \mapsto f(x, y)
\end{aligned}
\]
for ${x}$ of type ${\displaystyle X}$ and ${y}$ of type ${\displaystyle Y}$. We then also write
\[
    \text{curry}(f) = g    
\]

To illustrate currying, let's first introduce the concept of \textbf{types}.

In programming languages, expressions have values. We often say that an expression has a value of \texttt{1} or a value of \texttt{'a'}. Although they are stored in memory in binary format, they are conceptually distinct. To distinguish between them, we assign different types.

We refer to the value \texttt{1}'s type as \texttt{Int}, representing an integer, and the value \texttt{'a'}s type as \texttt{Char}, representing a character.

Thus, we establish a layer of abstraction on the concept of values, where different values have distinct properties based on their types.

In mathematical terms, types act as an equivalence relation on the set of values, inducing a partition of this set. Values of the same type form an equivalence class.

A function is used to transform one or more values into another value. Functions are also considered values and therefore have types.

For example, the addition function \texttt{add} takes two integer values and produces another integer value. Once we have defined the add function, we can call it like this:

% add code here
\begin{lstlisting}
        add 1 2
    =   3
\end{lstlisting}

Now, let's consider a question: what happens if we only provide one parameter to \texttt{add}? What is the result of \texttt{add 1}?

We can visualize it like this: consider add as a machine with two slots. If we provide values \texttt{1} and \texttt{2} to each slot respectively, the machine will produce the result \texttt{3}. Now, if only one slot is filled with \texttt{1}, what happens? Clearly, it's still a machine, but with only one slot filled. Therefore, \texttt{add 1} remains a function, but it only accepts one parameter. It returns the result of adding \texttt{1} to the provided parameter value. We can denote its type as:
\begin{lstlisting}
    add 1 :: Int -> Int
\end{lstlisting}

Let's reconsider the type of \texttt{add}. We observe that when \texttt{add} is given \texttt{1} as a parameter, it returns the function \texttt{add 1}. Therefore, we have an alternative perspective on \texttt{add}. It can be viewed as a machine with one slot: when provided with the parameter 1, it produces another machine with one slot, \texttt{add 1}. Hence, we can denote the type of add as:

\begin{lstlisting}
    add :: Int -> (Int -> Int)
\end{lstlisting}

For convenience in writing, we agree on right associativity for \texttt{->}, thus:

\begin{lstlisting}
    add :: Int -> Int -> Int
\end{lstlisting}

In fact, according to the notation provided above, Currying refers to a higher-order function as follows:

\begin{lstlisting}
    curry :: ((a, b) -> c) -> (a -> b -> c)
\end{lstlisting}

\noindent It transforms a function \texttt{f} into a function \texttt{g}, such that:
\begin{lstlisting}
    f :: ((a, b) -> c)
    g :: a -> b -> c
\end{lstlisting}

\noindent where \lstinline{g = curry f} and \lstinline{f = uncurry g}. This ensures that for any \texttt{x} and \texttt{y}, it satisfies:

\begin{lstlisting}
    f (x, y) = g x y
\end{lstlisting}

You might wonder what's the point of curring. 
Actually, curring brings much flexibility and makes it easier for composition: curried functions can be easily composed together to create more complex functions. 
By partially applying functions and combining them, one can create new functions without modifying existing ones, leading to more flexible and expressive expression.

Curring is a fundamental concept in the future development of so-called functional programming languages, which are derived from the $\lambda$-Calculus.
At that time, Currying had few applications in fields other than logic, but Curry's vision was ahead of its time. 
Today, decades later, the \textbf{Haskell} programming language based on Curring's functions has become one of the most popular programming languages in the world, widely used Used in various fields of engineering.
\begin{figure}
    \centering
    \includegraphics[width=0.5\textwidth]{figures/haskell.png}
    \caption{Haskell Programming Language}
    \label{fig:haskell}
\end{figure}